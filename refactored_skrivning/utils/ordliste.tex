%!TEX root = main.tex
\makenoidxglossaries

\newglossaryentry{smartpool}{
	name={Smartpool}, 
	description={Navnet på projektet og systemet.}
}

\newglossaryentry{dotcover}{
	name={DotCover}, 
	description={Lavet af JetBrains. dotCover er et .NET test-runner og code coverage værktøj - \url{https://www.jetbrains.com/dotcover/}}
}

\newglossaryentry{ef}{
	name={Entity Framework}, 
	description={Missing!}\todo{Skal skrives}
}

\newglossaryentry{windserver}{
	name={Connection Server}, 
	description={Missing!}\todo{Skal skrives}
}

\newglossaryentry{gui}{
    name={GUI}, 
    description={Graphical User Interface er en brugergrænseflade. Brugerens måde at tilgå systemet på}
}

\newglossaryentry{us}{
	name={User stories}, 
	description={User stories er en metode til at specificere de funktionelle krav til et system.}
}

\newglossaryentry{iosapp}{
	name={iOS-applikation}, 
	description={Et program der med et GUI tillader brugeren at bruge systemet fra en iPhone.}
}

\newglossaryentry{webapp}{
	name={Web-applikation}, 
	description={Et program der med et GUI tillader brugeren at bruge systemet fra en internet browser på alle platforme.}
}

\newglossaryentry{pcapp}{
	name={PC-applikation}, 
	description={Et program der med et GUI tillader brugeren at bruge systemet fra en Windows PC.}
}

\newglossaryentry{moscow}{
    name={MoSCoW}, 
    description={MoSCoW er en metode til at prioritere krav til ens system}
}

% muligvis ikke nødvendig
\newglossaryentry{user}{
		name={User},
		description={Bruger eller kunde. Brugeren der er oprettet i systemet eller brugeren, der har en pool.}
}
	
\newglossaryentry{admin}{
	name={Administrator},
	description={En ansat der kan tilgå databasen og direkte ændre på den i tilfælde af fejl.}
}

\newglossaryentry{tv}{
	name={Target-værdier},
	description={De ønskede værdier for de forskellige mål i poolen, som f.eks. temperatur.}
}

\newglossaryentry{mu}{
	name={MonitorUnit},
	description={Fysisk enhed med sensorer der installeres i poolen. Kan også referere til entitet på databasen, som indeholder målingerne fra den fysiske MoniterUnit}
}

\newglossaryentry{specinvite}{
	name={Spectate invitation},
	description={En måde hvorpå en bruger tillader en anden bruger at blive spectator af én af sine pools.}
}

\newglossaryentry{spectator}{
	name={Spectator},
	description={En bruger i systemet, som kan se en pool, men ikke ændre på den, da den tilhører en anden bruger.}
}

\newglossaryentry{nunit}{
	name={NUnit},
	description={Unittest framework til .NET - \url{http://www.nunit.org/}}
}

\newglossaryentry{nsub}{
	name={NSubstitute},
	description={Mocking framework til .NET - \url{http://nsubstitute.github.io/}}
}

\newglossaryentry{db}{
	name={Database},
	description={Relational database. En server der indeholder alt data i systemet.}
}

\newglossaryentry{pool}{
	name={Pool},
	description={Swimming pool hvor i produktet kan installeres.}
}

\newglossaryentry{Microsoft .NET}{
	name={Microsoft .NET},
	description={Microsoft's .NET-framework er blevet brugt i udviklingen af de meste af sourcekoden i projektet.}
}

\newglossaryentry{NUnit}{
	name={NUnit},
	description={Er et test framework som er anvendt til at skrive automatiske tests}
}

\newglossaryentry{NSubstitute}{
	name={NSubstitute},
	description={Er et mocking bibliotek der anvendes til at generere fakes til tests i NUnit}
}

\newglossaryentry{WPF}{
	name={WPF},
	description={WPF-framework'et er blevet brugt til udviklingen af systemets Windows brugergrænseflade.}
}

\newglossaryentry{ASP.NET MVC}{
	name={ASP.NET MVC},
	description={ASP.NET MVC er et framework, der er blevet brugt, til udviklingen af Web-applikationen.}
}

\newglossaryentry{NUnit Testframework}{
	name={NUnit Testframework},
	description={Et oplagt værktøj til softwaretest har været Nunit idet det bruges i vid udstrækning i forbindelse med kurset Software Test.}
}

\newglossaryentry{ADO.NET Entity}{
	name={ADO.NET Entity},
	description={ADO.NET Entity-framework'et er brugt til implementeringen af data-access laget. Framework'et er brugt for at muliggøre objektorienteret interaktion med systemets database.}
}

\newglossaryentry{UIKit}{
	name={UIKit},
	description={UIKit er et GUI-framework, udbudt af Apple, der er blevet brugt til udviklingen af brugergrænsefladen i iOS applikationen. UIKit udbydes i Objective-C og Swift. I projektet er MonoDevelop's C\# bridge af UIKit blevet brugt.
}
}


