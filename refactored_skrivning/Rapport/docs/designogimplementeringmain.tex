\chapter{Design og implementering}\label{cha:design}
%% indledning til kapitel skal være her...
I dette afsnit beskrives designet og implementering af software til \gls{smartpool} systemet. De forskellige ansvarsområder som applikation, forbindelse og data beskrives med indledende overvejelser. 
Designet og efterfølgende implementeringen beskrives med de to samme krav: 

\begin{itemize}
	\item Som bruger vil jeg kunne oprette mig i systemet, for at få adgang til systemet.
	\item Som bruger vil jeg kunne se de seneste sensor værdier for at kunne få et overblik over 
	poolens tilstand.
\end{itemize}

\section{Applikation}
Applikationerne i systemet er jævnfør arkitekturen lavet med MVP, og de respektive dele bliver beskrevet i rækkefølgen Model, Presenter og View. Lagets model- og præsentationslag er designet til at være portabelt, så forskellige platforme kunne dele denne kodebase. View-laget designes derimod seperat for hver platform. 

Det overordnede design i applikationslaget illustreres ved sekvensdiagrammet i figur~\ref{fig:application_sd}. Her fremgår kommunikationsmønsteret mellem model-, view- og presenter-klasserne i applikationslagets design.

\begin{figure}
	\centering
	\includegraphics[width=1.0\linewidth]{figs/design/application_sd}
	\caption{Kommunikation i applikationslaget}
	\label{fig:application_sd}
\end{figure}

Som følge systemets multi-lag arkitektur kunne applikationslaget samt de platform-specifikke view-implementeringer
designes, uafhængigt af hinanden, så længe protokolbestemmende interfaces blev specificeret undervejs i processen. Interfaces for både presenters og views, er defineret ud fra de user stories, der har krævet deres design.

\section{Connection}\label{sec:designconnection}

For at give systemet en øget sikkerhed, og bedre scalability, blev det besluttet at al business logic skulle varetages af en form for server. På denne måde indkaples essentielle dele af systemet, så brugere ikke har direkte adgang til systemkritiske elementer. Desuden kan der laves fremtidige ændringer i systemet, uden alle klienter nødvendigvis skal opdateres.

user sessions ??

Det blev yderligere besluttet at denne server skulle modtage data fra pools koblet til systemet. Da reele data rækker ud over dette projekts afgrænsning, blev det besluttet at lave en simulering af data, som foregår på serveren.

På baggrund af de valgte user stories ?? og design valg, blev der undersøgt hvilke teknologier, som kunne anvendes til at opnå det ønskede. Da der i et sideløbende netværksfag (IKN) ?? blev arbejdet meget med TCP sockets, virkede det som en oplagt mulighed, da dette netop blev anvendt til, at etablere en server og en klient.

% !TEX root = ../../I4PRJ, Grp3 - Rapport.tex
\subsubsection{Implementering}
Applikationslagets model- og presenterklasser er udviklet i C\#, i Microsoft Visual Studio. Klasserne er implementeret under følgende namespaces: Smartpool.Application.Model, Smartpool.Connection.Model og Smartpool.Application.Presentation. I dette afsnit beskrives de presenter-klasser, der blev introduceret i rapportens designafsnit. Herefter beskrives view-implementeringen i de platform-specifikke applikationer. For yderligere dokumentation af Smartpool.Application.Model og Smartpool.Application.Presentation henvises til dokumentationen.

\paragraph{SignUpViewController.cs}
Implementeringen af SignUpViewController-klassen bestod hovedsageligt i, at sammenbinde ISignUpViewController-interfacets metoder, med handlinger der skulle foretages i SignUpViewController-klassens ISignUpView. En af disse handlinger er oprettelsen af en bruger. Metoden SignUp, der kaldes når presenter-klassen modtager et ButtonPressed kald fra view'et, kan ses i listing~\ref{code:application_impl_signupmethod}. I metoden oprettes en besked, der sendes til serveren gennem den presenterens IClientMessenger. Når et svar modtages fra serveren, tjekker presenteren hvorvidt anmodningen var succesfuld. Herefter kaldes en metode i view'et, alt efter om anmodningen gik igennem eller ej. Hvis anmodningen blev afvist kaldes DisplayAlert i view'et.

\begin{lstlisting}[caption={SignUp()},label={code:application_impl_signupmethod}]
public void SignUp()
{
	var signUpRequest = new AddUserRequestMsg(User.Name, User.Email, User.Passwords[0]);
	var response = _clientMessenger.SendMessage(signUpRequest);
	var generalResponse = (GeneralResponseMsg) response;

	if (generalResponse.RequestExecutedSuccesfully)
		_view.SignUpAccepted();
	else
		_view.DisplayAlert("Invalid Sign Up", response.MessageInfo);
}
\end{lstlisting}

\paragraph{StatViewController.cs}
StatViewController-klassen håndterer indlæsning af pool-data til det tilknyttede view. I StatViewController-klassens ViewDidLoad, set i listing~\ref{code:application_impl_vdlstat}, bruges klassens PoolLoader til at indlæse pool-oplysninger i view'et.

\begin{lstlisting}[caption={ViewDidLoad() in StatViewController},label={code:application_impl_vdlstat}]
public void ViewDidLoad()
{
	// Load pools from server
	_loader.ReloadPools(_clientMessenger);

	// Load active pool info into text fields
	if (!_loader.PoolsAreAvailable())
		_view.DisplayAlert("No pools available", "You have not added any pools yet. Please go to 'Add Pool' to add a pool.");
	else
	{
		_view.SetAvailablePools(_session.Pools);
		_view.SetSelectedPoolIndex(_session.SelectedPoolIndex);
		LoadSensorData();
	}
}
\end{lstlisting}

ViewDidLoad-metoden benytter sig af en private hjælpemetode, LoadSensorData. Metoden, som ses i listing~\ref{code:application_impl_statlsd}, kalder view interface-metoden DisplaySensorData og bruger klassens PoolLoader, til at indlæse måleværdier i view'et.

\begin{lstlisting}[caption={LoadSensorData() in StatViewController},label={code:application_impl_statlsd}]
private void LoadSensorData()
{
	// Loads current sensor data into the view
	_view.DisplaySensorData(_loader.GetCurrentDataFromPool(_clientMessenger));
}
\end{lstlisting}
\section{View}
\subsubsection{Implementering}
Windows implementeres jævnfør designet ved at lave vinduer eller views, som implementerer view interfaces fra design af presentation-laget.

På figur~\ref{fig:wincreateuserandwinstatviewview} ses de klassediagrammet for de to views der præsenterer muligheden for at løse de to user stories, der er taget udgangspunkt i i starten af design og implementerings afsnittet.

Resultatet af implementeringen af WinStatView ses på følgende figur.

\begin{figure}
\centering
\includegraphics[width=0.5\linewidth]{figs/implementering/winstatview}
\caption{WinStatView resultat}
\label{fig:winstatview}
\end{figure}

I følgende kode er der udsnit fra view'et der præsenterer historisk data for brugeren. 

\begin{lstlisting}[caption=DisplayGraph, label=DisplayGraph]
private void DisplayGraph(Canvas historyCanvas, List<double> history, bool isPhOrChlorine)
{
...
var pointHeight = ((upperBound - history[i])) / (upperBound - lowerBound) * canvasHeight;
var pointWidth = (canvasWidth/(_pointsOnGraphs - 1))*i;
...
\end{lstlisting}

Koden er medbragt for at fremvise et view med mere dybde end de to på figur~\ref{fig:wincreateuserandwinstatviewview}.
Funktionen DisplayGraph() modtager historisk data og tegner en graf. 
På linje 9 og 10 udregnes punkterne på grafens højde relativt til størrelsen af grafen. At de udregnes relativt, gør at, grafen skalerer perfekt, når vinduet skaleres. 
Hver punkt i datamængden funktionen modtager tilføjes til et canvas, og linjes tegnes imellem.
Resultatet af DisplayGraph ses på følgende figur.

\begin{figure}
\centering
\includegraphics[width=0.4\linewidth]{figs/implementering/displaygraph}
\caption{Graf fra WinHistoryView}
\label{fig:displaygraph}
\end{figure}

For mere information se dokumentationen afsnit Windows Implementering.
\subsubsection{Implementering}
Windows implementeres jævnfør designet ved at lave vinduer eller views, som implementerer view interfaces fra design af presentation-laget.

På figur~\ref{fig:wincreateuserandwinstatviewview} ses de klassediagrammet for de to views der præsenterer muligheden for at løse de to user stories, der er taget udgangspunkt i i starten af design og implementerings afsnittet.

Resultatet af implementeringen af WinStatView ses på følgende figur.

\begin{figure}
\centering
\includegraphics[width=0.5\linewidth]{figs/implementering/winstatview}
\caption{WinStatView resultat}
\label{fig:winstatview}
\end{figure}

I følgende kode er der udsnit fra view'et der præsenterer historisk data for brugeren. 

\begin{lstlisting}[caption=DisplayGraph, label=DisplayGraph]
private void DisplayGraph(Canvas historyCanvas, List<double> history, bool isPhOrChlorine)
{
...
var pointHeight = ((upperBound - history[i])) / (upperBound - lowerBound) * canvasHeight;
var pointWidth = (canvasWidth/(_pointsOnGraphs - 1))*i;
...
\end{lstlisting}

Koden er medbragt for at fremvise et view med mere dybde end de to på figur~\ref{fig:wincreateuserandwinstatviewview}.
Funktionen DisplayGraph() modtager historisk data og tegner en graf. 
På linje 9 og 10 udregnes punkterne på grafens højde relativt til størrelsen af grafen. At de udregnes relativt, gør at, grafen skalerer perfekt, når vinduet skaleres. 
Hver punkt i datamængden funktionen modtager tilføjes til et canvas, og linjes tegnes imellem.
Resultatet af DisplayGraph ses på følgende figur.

\begin{figure}
\centering
\includegraphics[width=0.4\linewidth]{figs/implementering/displaygraph}
\caption{Graf fra WinHistoryView}
\label{fig:displaygraph}
\end{figure}

For mere information se dokumentationen afsnit Windows Implementering.
% !TEX root = ../../../I4PRJ, Grp3 - Rapport.tex
\subsection{iOS GUI}
Til iOS er der udviklet et view-lag med UIKit og C\#. Det findes i projektet Smartpool.Application.iOS.
Den visuelle implementering foretages med storyboards i Interface Builder, som er en del af Xcode-udviklingsværktøjet.

\subsubsection{Design}
I iOS applikationen designes view-klasser, der implementerer view-interfacet defineret i præsentationslaget. Brugergrænseflader på iOS platformen designes med frameworket UIKit, hvor hvert view har en tilhørende controller-klasse. Det skulle der tages højde for i designet, hvor ønsket var, at iOS designet udelukkende skulle bestå af views. Løsningen blev, at bruge UIKit controller-klasser blev brugt som broer, i mellem Smartpools præsentationslag og UIKit's view-lag. Designet er derfor lavet således, at controller-klasserne på iOS implementerer view-interfacet fra Smartpools præsentationslag.

SignUpViewBridge og StatViewBridge er de to klasser i iOS-applikationen, som implementerer hhv. ISignUpView og IStatView. Designet af de to klasser fremgår af figur~\ref{fig:ios_viewbridges}.

\begin{figure}
	\centering
	\includegraphics[width=0.7\linewidth]{figs/design/ios_viewbridges}
	\caption{SignUpViewBridge og StatViewBridge}
	\label{fig:ios_viewbridges}
\end{figure}

Udover at implementere view-interfacet, indeholder begge klasser en række UIKit elementer. Valget af user interface elementer er taget ud fra de user stories der krævede deres design.

For yderligere forklaring se dokumentation afsnit Applikationslaget under Design.
% !TEX root = ../../../I4PRJ, Grp3 - Rapport.tex
\subsection{iOS GUI}
Til iOS er der udviklet et view-lag med UIKit og C\#. Det findes i projektet Smartpool.Application.iOS.
Den visuelle implementering foretages med storyboards i Interface Builder, som er en del af Xcode-udviklingsværktøjet.

\subsubsection{Design}
I iOS applikationen designes view-klasser, der implementerer view-interfacet defineret i præsentationslaget. Brugergrænseflader på iOS platformen designes med frameworket UIKit, hvor hvert view har en tilhørende controller-klasse. Det skulle der tages højde for i designet, hvor ønsket var, at iOS designet udelukkende skulle bestå af views. Løsningen blev, at bruge UIKit controller-klasser blev brugt som broer, i mellem Smartpools præsentationslag og UIKit's view-lag. Designet er derfor lavet således, at controller-klasserne på iOS implementerer view-interfacet fra Smartpools præsentationslag.

SignUpViewBridge og StatViewBridge er de to klasser i iOS-applikationen, som implementerer hhv. ISignUpView og IStatView. Designet af de to klasser fremgår af figur~\ref{fig:ios_viewbridges}.

\begin{figure}
	\centering
	\includegraphics[width=0.7\linewidth]{figs/design/ios_viewbridges}
	\caption{SignUpViewBridge og StatViewBridge}
	\label{fig:ios_viewbridges}
\end{figure}

Udover at implementere view-interfacet, indeholder begge klasser en række UIKit elementer. Valget af user interface elementer er taget ud fra de user stories der krævede deres design.

For yderligere forklaring se dokumentation afsnit Applikationslaget under Design.
% !TEX root = ../../I4PRJ, Grp3 - Dokumentation.tex
\subsubsection{Implementering}
Implementeringen af den grafiske brugergrænseflade til Web-applikationen beskrives i dette afsnit. Kildekoden ligger i Application.Web, hvori projektet af samme navn forefindes. Fælles for alle web views er, at de er implementeret med HTML. HTML er et markup language til hjemmesider. Hjemmesidens grafiske layout laves vha. CSS som beskriver, hvordan man vil have HTML vist på sin hjemmeside. Ydermere er der brugt javascript til fra client-side at kommunikere med brugeren. Slutteligt er der brugt razor, som er et server-side markup language. 

\paragraph{LoginView}
LoginView ses på figuren nedenfor:

\begin{figure}
	\centering
	\includegraphics[width=1.0\linewidth]{figs/implementering/web_login}
	\caption{Web LoginView}
	\label{fig:webloginview}
\end{figure}

View grafikken er implementeret jævnfør det grafiske design. AccountController som implementerer ILoginView interfacet, har funktionen InitateController. Web-applikationens implementering af funktionen ses nedenfor:

\begin{lstlisting}[caption=InitiateController, label=code:InitaiteController]
private void InitiateController()
        {
            Controller = new LoginViewController(this, new ClientMessenger(new SynchronousSocketClient("93.166.226.201")));
            Controller.ViewDidLoad();
        }
\end{lstlisting} 

Funktionen oprtter en ny loginViewController, og giver den en ClientMessenger med sammen med en IP-adresse, hvorefter funktionen ViewDidLoad kaldes.

For at logge ind på serveren laves en 'Task' login.

\begin{lstlisting}[caption=Login, label=code:Login]
 // POST: /Account/Login
        [HttpPost]
        [AllowAnonymous]
        [ValidateAntiForgeryToken]
        public async Task<ActionResult> Login(LoginViewModel model, string returnUrl)
        {
            var loginController = Controller as ILoginViewController;
            loginController.DidChangeEmailText(model.Email);
            loginController.DidChangePasswordText(model.Password);
            loginController.ButtonPressed(LoginViewButton.LoginButton);

            _returnUrl = returnUrl;

            return View(model);
        }
\end{lstlisting} 

Funktionen sender request til server ved Button.Pressed. Funktionen LoginAccepted kaldes når serveren svarer tilbage.

\begin{lstlisting}[caption=LoginAccepted, label=code:LoginAccepted]
public void LoginAccepted()
        {
            ActionInvoker.InvokeAction(ControllerContext, "RedirectLogin");
        }
\end{lstlisting}

LoginAccepted 'Invoker' et ActionResult RedirectLogin.
 
 \begin{lstlisting}[caption=Redirect Login, label=code:redirectlogin]

        [AllowAnonymous]
        public ActionResult RedirectLogin()
        {
            return RedirectToLocal(_returnUrl);
        }
        }
\end{lstlisting}

RedirectLogin redirects til en anden side på serveren når login accepteres.

 
\paragraph{SignUpView}\label{sec: signupview}
SignUpView ses på figuren nedenfor:

\begin{figure}
	\centering
	\includegraphics[width=1.0\linewidth]{figs/implementering/web_signupview}
	\caption{Web SignUpView}
	\label{fig:websignupview}
\end{figure}

View grafikken er implementeret jævnfør det grafiske design. AccountController implementerer endnu ikke ISignUpView. Den ene sprint foretaget på Web, havde primært til formål at teste arkitekturen. Grunden tidspres er det resterende website blevet nedprioriteret. Det gælder ligeledes for de resterende views.

\paragraph{AddPoolView}
AddPoolView ses på figuren nedenfor:

\begin{figure}
	\centering
	\includegraphics[width=1.0\linewidth]{figs/implementering/web_addpoolview}
	\caption{Web AddPoolView}
	\label{fig:webaddpoolview}
\end{figure}

Brugeren kan vælge enten at indtaste volumen eller dimensioner, ved at vælge på radioknapperne. Når den ene er valgt skal den anden mulighed ikke være tilgængeligt. Det håndterer scriptet enableTxtBox. Er brugeren begyndt at indtaste volumen, men ombestemmer sig, så skal tekstfeltet tømmes, det håndteres af scriptet clearText. Der er ikke muligt for brugeren at indtaste andet end tal, det håndterer scriptet isNumber. Alle scripts ses i kodeudsnittet nedenfor.

\begin{lstlisting}[caption=AddPoolScripts, label=code:scripts]
 <script>
	function enableTxtBox1()
	{
         	document.getElementById("text1").disabled = !document.getElementById("radio1").checked;
         	document.getElementById("text2").disabled = document.getElementById("radio1").checked;
         	document.getElementById("text3").disabled = document.getElementById("radio1").checked;
         	document.getElementById("text4").disabled = document.getElementById("radio1").checked;
    }
</script>

<script>
	function clearText()
	{
		if(!document.getElementById("radio1").checked) text1.value="";
		if(document.getElementById("radio1").checked) {
         	text2.value="";
         	text3.value="";
         	text4.value="";
         }
    }
</script>

<script>
function isNumber(evt) {
    evt = (evt) ? evt : window.event;
    var charCode = (evt.which) ? evt.which : evt.keyCode;
    if (charCode > 31 && (charCode < 48 || charCode > 57)) {
        return false;
    }
    return true;
}
</script>

\end{lstlisting} 


View grafikken er implementeret jævnfør det grafiske design. AccountController implementerer endnu ikke IAddPoolView. Se \ref{sec: signupview} for yderligere forklaring.

\paragraph{EditUserView}
AddPoolView ses på figuren nedenfor:

\begin{figure}
	\centering
	\includegraphics[width=1.0\linewidth]{figs/implementering/web_edituserview}
	\caption{Web EditUserView}
	\label{fig:webedituserview}
\end{figure}

View grafikken er implementeret jævnfør det grafiske design. AccountController implementerer endnu ikke IEditUserView. Se \ref{sec: signupview} for yderligere forklaring.

\paragraph{EditPoolView}
Dette view er endnu ikke implementeret. Se \ref{sec: signupview} for yderligere forklaring.

\paragraph{StatView}
StatView ses på figuren nedenfor:

\begin{figure}
	\centering
	\includegraphics[width=1.0\linewidth]{figs/implementering/web_statview}
	\caption{Web StatView}
	\label{fig:webstatview}
\end{figure}

View grafikken er implementeret jævnfør det grafiske design. AccountController implementerer endnu ikke IStatView. Se \ref{sec: signupview} for yderligere forklaring.

\paragraph{HistoryView}

Dette view er endnu ikke implementeret. Se \ref{sec: signupview} for yderligere forklaring.

% !TEX root = ../../I4PRJ, Grp3 - Dokumentation.tex
\subsubsection{Implementering}
Implementeringen af den grafiske brugergrænseflade til Web-applikationen beskrives i dette afsnit. Kildekoden ligger i Application.Web, hvori projektet af samme navn forefindes. Fælles for alle web views er, at de er implementeret med HTML. HTML er et markup language til hjemmesider. Hjemmesidens grafiske layout laves vha. CSS som beskriver, hvordan man vil have HTML vist på sin hjemmeside. Ydermere er der brugt javascript til fra client-side at kommunikere med brugeren. Slutteligt er der brugt razor, som er et server-side markup language. 

\paragraph{LoginView}
LoginView ses på figuren nedenfor:

\begin{figure}
	\centering
	\includegraphics[width=1.0\linewidth]{figs/implementering/web_login}
	\caption{Web LoginView}
	\label{fig:webloginview}
\end{figure}

View grafikken er implementeret jævnfør det grafiske design. AccountController som implementerer ILoginView interfacet, har funktionen InitateController. Web-applikationens implementering af funktionen ses nedenfor:

\begin{lstlisting}[caption=InitiateController, label=code:InitaiteController]
private void InitiateController()
        {
            Controller = new LoginViewController(this, new ClientMessenger(new SynchronousSocketClient("93.166.226.201")));
            Controller.ViewDidLoad();
        }
\end{lstlisting} 

Funktionen oprtter en ny loginViewController, og giver den en ClientMessenger med sammen med en IP-adresse, hvorefter funktionen ViewDidLoad kaldes.

For at logge ind på serveren laves en 'Task' login.

\begin{lstlisting}[caption=Login, label=code:Login]
 // POST: /Account/Login
        [HttpPost]
        [AllowAnonymous]
        [ValidateAntiForgeryToken]
        public async Task<ActionResult> Login(LoginViewModel model, string returnUrl)
        {
            var loginController = Controller as ILoginViewController;
            loginController.DidChangeEmailText(model.Email);
            loginController.DidChangePasswordText(model.Password);
            loginController.ButtonPressed(LoginViewButton.LoginButton);

            _returnUrl = returnUrl;

            return View(model);
        }
\end{lstlisting} 

Funktionen sender request til server ved Button.Pressed. Funktionen LoginAccepted kaldes når serveren svarer tilbage.

\begin{lstlisting}[caption=LoginAccepted, label=code:LoginAccepted]
public void LoginAccepted()
        {
            ActionInvoker.InvokeAction(ControllerContext, "RedirectLogin");
        }
\end{lstlisting}

LoginAccepted 'Invoker' et ActionResult RedirectLogin.
 
 \begin{lstlisting}[caption=Redirect Login, label=code:redirectlogin]

        [AllowAnonymous]
        public ActionResult RedirectLogin()
        {
            return RedirectToLocal(_returnUrl);
        }
        }
\end{lstlisting}

RedirectLogin redirects til en anden side på serveren når login accepteres.

 
\paragraph{SignUpView}\label{sec: signupview}
SignUpView ses på figuren nedenfor:

\begin{figure}
	\centering
	\includegraphics[width=1.0\linewidth]{figs/implementering/web_signupview}
	\caption{Web SignUpView}
	\label{fig:websignupview}
\end{figure}

View grafikken er implementeret jævnfør det grafiske design. AccountController implementerer endnu ikke ISignUpView. Den ene sprint foretaget på Web, havde primært til formål at teste arkitekturen. Grunden tidspres er det resterende website blevet nedprioriteret. Det gælder ligeledes for de resterende views.

\paragraph{AddPoolView}
AddPoolView ses på figuren nedenfor:

\begin{figure}
	\centering
	\includegraphics[width=1.0\linewidth]{figs/implementering/web_addpoolview}
	\caption{Web AddPoolView}
	\label{fig:webaddpoolview}
\end{figure}

Brugeren kan vælge enten at indtaste volumen eller dimensioner, ved at vælge på radioknapperne. Når den ene er valgt skal den anden mulighed ikke være tilgængeligt. Det håndterer scriptet enableTxtBox. Er brugeren begyndt at indtaste volumen, men ombestemmer sig, så skal tekstfeltet tømmes, det håndteres af scriptet clearText. Der er ikke muligt for brugeren at indtaste andet end tal, det håndterer scriptet isNumber. Alle scripts ses i kodeudsnittet nedenfor.

\begin{lstlisting}[caption=AddPoolScripts, label=code:scripts]
 <script>
	function enableTxtBox1()
	{
         	document.getElementById("text1").disabled = !document.getElementById("radio1").checked;
         	document.getElementById("text2").disabled = document.getElementById("radio1").checked;
         	document.getElementById("text3").disabled = document.getElementById("radio1").checked;
         	document.getElementById("text4").disabled = document.getElementById("radio1").checked;
    }
</script>

<script>
	function clearText()
	{
		if(!document.getElementById("radio1").checked) text1.value="";
		if(document.getElementById("radio1").checked) {
         	text2.value="";
         	text3.value="";
         	text4.value="";
         }
    }
</script>

<script>
function isNumber(evt) {
    evt = (evt) ? evt : window.event;
    var charCode = (evt.which) ? evt.which : evt.keyCode;
    if (charCode > 31 && (charCode < 48 || charCode > 57)) {
        return false;
    }
    return true;
}
</script>

\end{lstlisting} 


View grafikken er implementeret jævnfør det grafiske design. AccountController implementerer endnu ikke IAddPoolView. Se \ref{sec: signupview} for yderligere forklaring.

\paragraph{EditUserView}
AddPoolView ses på figuren nedenfor:

\begin{figure}
	\centering
	\includegraphics[width=1.0\linewidth]{figs/implementering/web_edituserview}
	\caption{Web EditUserView}
	\label{fig:webedituserview}
\end{figure}

View grafikken er implementeret jævnfør det grafiske design. AccountController implementerer endnu ikke IEditUserView. Se \ref{sec: signupview} for yderligere forklaring.

\paragraph{EditPoolView}
Dette view er endnu ikke implementeret. Se \ref{sec: signupview} for yderligere forklaring.

\paragraph{StatView}
StatView ses på figuren nedenfor:

\begin{figure}
	\centering
	\includegraphics[width=1.0\linewidth]{figs/implementering/web_statview}
	\caption{Web StatView}
	\label{fig:webstatview}
\end{figure}

View grafikken er implementeret jævnfør det grafiske design. AccountController implementerer endnu ikke IStatView. Se \ref{sec: signupview} for yderligere forklaring.

\paragraph{HistoryView}

Dette view er endnu ikke implementeret. Se \ref{sec: signupview} for yderligere forklaring.


\section{Forbindelse}
\section{Connection}\label{sec:designconnection}

For at give systemet en øget sikkerhed, og bedre scalability, blev det besluttet at al business logic skulle varetages af en form for server. På denne måde indkaples essentielle dele af systemet, så brugere ikke har direkte adgang til systemkritiske elementer. Desuden kan der laves fremtidige ændringer i systemet, uden alle klienter nødvendigvis skal opdateres.

user sessions ??

Det blev yderligere besluttet at denne server skulle modtage data fra pools koblet til systemet. Da reele data rækker ud over dette projekts afgrænsning, blev det besluttet at lave en simulering af data, som foregår på serveren.

På baggrund af de valgte user stories ?? og design valg, blev der undersøgt hvilke teknologier, som kunne anvendes til at opnå det ønskede. Da der i et sideløbende netværksfag (IKN) ?? blev arbejdet meget med TCP sockets, virkede det som en oplagt mulighed, da dette netop blev anvendt til, at etablere en server og en klient.

\subsection{Implementering}
I følgende afsnit vil den overordnede implementering af de enkelte dele blive beskrevet.
\subsubsection{Overordnet connection struktur}
\textit{Klient} delen består af et model projekt som er generelt for alle platforme, samt et klient projekt som er specifikt for hver platform. Dette er gjort for at gøre så meget som muligt anvendeligt på alle platforme. I model projektet er desuden defineret en række besked objekter, som anvendes ved kommunikation mellem klient og server.

\textit{Server} delen består af en række systemer som tilsammen udgør en samlet server udviklet til at køre på en windows pc. Al kommunikation til databasen foregår fra server delen.

\subsubsection{Klient}
Klienten modtager besked objekter fra applikations laget, og omdanner disse til en streng vha. Json serializering. Strengen bliver sendt til server delen gennem en socket klient der passer til den pågældende platform.
Klienten modtager derefter et svar fra serveren, i form af et serializeret besked objekt, som bliver deserializeret til basis besked klassen. Denne indeholder en besked type, og klienten kan derefter deserializere den modtagne streng til det korrekte besked objekt. Derefter bliver objektet sendt videre, tilbage til applikations laget. 

\subsubsection{Server}
Vedligeholder følgende funktioner i systemet
\begin{itemize}
	\item Modtage, behandle og svare på requests fra klient delen
	\item Varetage user sessions
	\item Kommunikere med databasen via metoder i database delen
	\item Simulere pool data
\end{itemize}

\subsubsection{Modtage, behandle og svare på requests fra klient delen}
Serveren modtager via en Asyncronous Socket Client (ASC) en streng fra klient delen. Denne bliver, som i klienten, lavet til et basis besked objekt. Denne bliver derefter behandlet vha. en switchcase, hvor der reageres på hvilken message type der er blevet sendt. Dette foregår i en ResponseManager klasse, og denne vil, i nogle situationer, være i stand til selv at udføre den kaldte request. Det gør den ved f.eks. at lave et kald til databasen og returnere svaret derfra. 
I andre situationer, bliver kaldet sendt videre til en sub handler, som f.eks. TokenMsgResponse, der tager sig af alle requests, som kræver at brugeren er logget ind. Dette er lavet således at ResponseManager checker, via databasen, om brugeren er logget ind, og hvis dette er tilfældet, bliver beskeden sendt videre til TokenMsgResponse. TokenMsgResponse behøver dermed ikke selv at kontrollere om brugeren er logget ind.
Hvis beskedtypen ikke genkendes, sendes svar tilbage til klienten om dette.

\subsubsection{User sessions}
For at holde styr på hvilke brugere som er logget ind, er der udviklet et Token system. Dette system giver en øget sikkerhed, ved at brugeren kun sender password en enkelt gang, og det behøver derfor heller ikke at blive gemt i klienten. Desuden bliver der færre kald til databasen, da efterfølgende requests ikke behøver at kontrollere brugerens password via databasen.
Token systemet virker ved at en bruger, ved login, får tilknyttet en streng af tilfældige karakterer. Brugernavnet bliver, sammen med strengen af karakterer og et timestamp, gemt i en klasse der hedder Token. Alle disse tokens bliver så vedligeholdt i en TokenKeeper. Når en bruger efterfølgende laver en request til serveren, sender klienten både username og token strengen med i sin request. Serveren kontrollerer derefter om dette stemmer overens med de data som ligger i TokenKeeperen, samt om det gemte timestamp er ældre end systemets valgte sessions tid.
En bruger kan godt være logget ind på flere enheder på samme tid. I det tilfælde vil hver enhed få tildelt en token streng, og dermed har de ikke indflydelse på hinandens session.

\subsubsection{Kommunikere med databasen via metoder i database delen}
Serveren kommunikerer via tilgængelige metoder i systemets dataaccess layer ??

\subsubsection{Simulerering af pool data}
Da Smartpool systemet ikke anvender reele data, er der udviklet en FakePool klasse til at simulere dette. Denne opretter en af hver type sensor. Hver sensor bliver initieret med en værdi, der ligger indenfor et realistisk område, for den pågældende sensor type. 
Denne værdi opdateres med et angivet internal, hvorefter værdierne gemmes i databasen. Værdi opdateringen foregår med små tilfældigt genererede ændringer. Disse er yderligere begrænset af en minimum og maximum grænse, specificeret i en SensorValueAuthenticator klasse.  


\section{Database og Data-access lag}\label{sec:designdatabase}
I nedenstående afsnit beskrives designet og implementeringen af database og data access laget.

I følgende liste opridses de fire primære elementer i databasen.

\begin{itemize}
	\item User - En User er en entitet i databasen, som indeholder information om en af systemets \textit{brugere}.
	\item Pool - En Pool er også en entitet i databasen. Alle Pool-entiteter er \textit{ejet} af en User.
	\item Data-entiteten og de specifikke datatype-entiteter - Typer af sensordata, herunder klor, pH, temperatur og luftfugtighed.
\end{itemize}
\section{Connection}\label{sec:designconnection}

For at give systemet en øget sikkerhed, og bedre scalability, blev det besluttet at al business logic skulle varetages af en form for server. På denne måde indkaples essentielle dele af systemet, så brugere ikke har direkte adgang til systemkritiske elementer. Desuden kan der laves fremtidige ændringer i systemet, uden alle klienter nødvendigvis skal opdateres.

user sessions ??

Det blev yderligere besluttet at denne server skulle modtage data fra pools koblet til systemet. Da reele data rækker ud over dette projekts afgrænsning, blev det besluttet at lave en simulering af data, som foregår på serveren.

På baggrund af de valgte user stories ?? og design valg, blev der undersøgt hvilke teknologier, som kunne anvendes til at opnå det ønskede. Da der i et sideløbende netværksfag (IKN) ?? blev arbejdet meget med TCP sockets, virkede det som en oplagt mulighed, da dette netop blev anvendt til, at etablere en server og en klient.

\section{Database og Data Access Layer}

Der er med udgangspunkt i designovervejelserne i afsnit~\ref{sec:designdatabase} implementeret et fungerende data-access layer med tilhørende database.

Databasen er implementeret med en Model First tilgang \cite{microsoftdatadevelopercenter2016}. Det vil sige at der opsættes en model (ER diagram) for databasen i Visual Studio, hvorefter der genereres et SQL script der kan køres mod den specifikke database. Scriptet køres mod en tom database, hvor de opstillede entities genereres som tabeller.

Data entiteten som ses på figur~\ref{fig:databaseERD_final_uml} da bruges som en klasse, se klassedigram på figur~\ref{fig:efGeneratedData}. De forskellige datatyper, pH, chlorine, temperature og humidity figurerer som lister (ICollections) i Data klassen. Disse lister er i koden angivet som virtual. Dette er for at gøre lazy loading muligt.

\begin{figure}
\centering
\includegraphics[width=0.5\linewidth]{figs/implementering/efGeneratedData.PNG}
\caption{Data klassen - Genereret fra Entity Model}
\label{fig:efGeneratedData}
\end{figure}

\subsection{Implementering af data-access layer}

\subsubsection{Træk pooldata ud}

\begin{lstlisting}[caption= GetChlorineData method, label=code:getChlorineData]

public List<Tuple<SensorTypes, double>> GetChlorineValues(string poolOwnerEmail, string poolName, int daysToGoBack)
{
double days = System.Convert.ToDouble(daysToGoBack);
string now = DateTime.UtcNow.ToString("G");
string start = DateTime.Parse(now).AddDays(-days).ToString("G");

using (var db = new DatabaseContext())
{   
DateTime startTime = DateTime.ParseExact(start, "dd/MM/yyyy HH:mm:ss", System.Globalization.CultureInfo.InvariantCulture);
DateTime endTime = DateTime.ParseExact(now, "dd/MM/yyyy HH:mm:ss", System.Globalization.CultureInfo.InvariantCulture);

var chlorineDataQuery = from chlorine in db.ChlorineSet
where chlorine.Data.Pool.Name == poolName && chlorine.Data.Pool.User.Email == poolOwnerEmail
select chlorine;

List<Tuple<SensorTypes, double>> chlorineTuples = new List<Tuple<SensorTypes, double>>();

foreach (var chlorine in chlorineDataQuery)
{
if(DateTime.Parse(chlorine.Data.Timestamp).CompareTo(endTime) < 0 ||
DateTime.Parse(chlorine.Data.Timestamp).CompareTo(startTime) > 0)
{
chlorineTuples.Add(new Tuple<SensorTypes, double>(SensorTypes.Chlorine, chlorine.Value));
}
}

return chlorineTuples;
}
}
\end{lstlisting}

\subsubsection{Tilføj user}

\begin{lstlisting}[]
public bool AddUser(string fullname, string email, string password)
{

	if (IsEmailInUse(email)) return false;

	User user;

	if (!ValidateName(fullname)) return false;

	string[] names = fullname.Split(' ');

	if (names.Length <= 2)
	{
		user = new User() { Firstname = names[0], Lastname = names[1], Email = email, Password = password };
	}
	else
	{
		user = new User() { Firstname = names[0], Middelname = names[1], Lastname = names[2], Email = email, Password = password };
	}

	using (var db = new DatabaseContext())
	{
		db.UserSet.Add(user);
		db.SaveChanges();
	}

	return true;
}
\end{lstlisting}

