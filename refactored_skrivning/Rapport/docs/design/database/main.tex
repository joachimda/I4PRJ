\subsection{Indledende designovervejelser}\label{sec:designdatabase}

Til projektet er der udviklet en database som står for at gemme alle brugerdata, herunder brugere, brugernes pools samt alle sensormålinger der foretages på disse pools.

Den endelige database er udviklet med Entity Frameworket, med en \textit{Model First} tilgang. Dette er valgt efter mange overvejelser og flere forsøg med forskellige udviklingsmetoder. Dette er beskrevet i projekt dokumentationen i afnsittet om \todo{reference til dokumentationen}. I de nedenstående afsnit beskrives designet og implementeringen af både databasen og projektets data-access lag. Beskrivelserne tager udgangspunkt i de to primære user stories. Data-access laget er som helhed omfattende at skulle beskrive. Derfor findes der yderligere information om den resterende funktionalitet og design i projektets dokumentation.

Databasen og data-access laget er udviklet sideløbende, og har gennemgået 3 primære udviklingsfaser. Første fase med User, derefter Pool og til sidst Data. Dette afspejles ved, at der i projektet er arbejdet i iterationer, hvor der blev udviklet et funktionelt produkt for hver iteration.

Da databasen er udviklet med \textit{Object Relational Mapperen}, Entity Framework, som gør det lettere at arbejde med rå data, ved at give mulighed for at behandle dem som data objekter\todo{reference}.

Data access laget er udviklet specifikt til Smartpool databasen, og er designet med henblik på usability og robustness. Denne software komponent er testet grundigt, og let at bruge for klienter. De primære klasser og interfaces i data-access laget er følgende:

\begin{itemize}
	\item \textbf{SmartpoolDB som implementerer ISmartpoolDB.}\\
	Eksponerer et interface for, \gls{windserver}.
	\item \textbf{UserAccess som implementerer IUserAccess.}\\
	Står for al database adgang og database forespørgsler i forhold til systemets brugere samt brugerinformation.
	\item \textbf{PoolAccess som implementerer IPoolAccess.}\\
	Som det ovenstående står denne klasse for al database adgang og database forespørgsler i forhold til systemets pools.
	\item \textbf{DataAccess som implementerer IDataAccess.}\\
	Klassen data-access er på samme måde bindeleddet til al måledata, der foretages på pools.
\end{itemize}

De fulde klassediagram for data-access laget findes i projektdokumentationens Appendix A.

De data entiteter, der ses på figur~\ref{fig:databaseERD_final_uml}, findes også som C\# klasser. Det er dem, der udgør database-objekterne. Som det ses på figuren, har entiteten \textit{Data} en Timestamp property, samt associationer til alle sensor-data typerne. Sensor-data typerne holder selve måleværdierne. Når der laves en forespørgsel i databasen på måledata, bruges de enkelte sensorentiteters fremmednøgle, \textit{DataId} til at finde frem til deres egentlige Timestamp. Grunden til dette gøres er for at undgå redundant data i form af Timestamp på hver af sensordata entiteterne.

\subsection{Udvalgte User Stories}
Med udgangspunkt i de to følgende User Stories, beskrives database designet og data-acces laget. 

\begin{itemize}
	\item \textit{"Som bruger vil jeg kunne oprette mig, for at få adgang til systemet"}
	\item \textit{"Som bruger vil jeg kunne se de seneste sensorværdier for at kunne få et overblik over poolens tilstand"}
\end{itemize}

\subsection{User story - Oprettelse af en bruger i systemet}

Ved at tage udgangspunkt i projektets domæneanalyse, blev det gjort klart hvilke attributter, en Bruger/User af systemet skulle have. Disse attributter er vist på figur~\ref{fig:database_model_1}.

\begin{figure}[h]
\centering
\includegraphics[width=0.2\linewidth]{figs/database/database_model_1.png}
\caption{User entiteten}
\label{fig:database_model_1}
\end{figure}

Ud fra denne entitet laves et SQL script som kan køres på databasen. Tabellen som vises på figur~\ref{fig:usertable} er da oprettet i databasen gennem Entity Frameworket.

\begin{figure}[h]
\centering
\includegraphics[width=0.5\linewidth]{figs/database/usertable}
\caption{User entitens tabel i databasen}
\label{fig:usertable}
\end{figure}

Med en funktionel database er det nu oplagt at begynde på designet af data-access laget. Der ønskes herudover funktionalitet til at tage brugere ud af databasen. I forbindelse med udarbejdelsen af disse backendfunktonaliter blev \textit{Test Driven Development} brugt, hvor der blev identificeret testcases og testscenarier. Denne måde at arbejde på har også sat sit præg på data-access lagets test coverage procent, som set på figur~\ref{fig:coverage}.

\begin{figure}[h]
\centering
\includegraphics[width=0.75\linewidth]{figs/database/coverage.png}
\caption{Test coverage for Smartpool systemet}
\label{fig:coverage}
\end{figure}

Som testene blev skrevet, blev tilsvarende business-logic og funktionalitet implementeret.
Det har hele tiden været tanken at data-access laget skulle være nemt at bruge hos \gls{windserver}. Derfor er løsningen designet med interfacet, ISmartpoolDB, som kan ses på figur~\ref{fig:classISmartpoolDB}.

\begin{figure}[h]
\centering
\includegraphics[width=0.3\linewidth]{figs/database/classISmartpoolDB}
\caption{Interfacet ISmartpoolDB}
\label{fig:classISmartpoolDB}
\end{figure}

På denne måde er systemet åbent for udvidelse, da der let kan tilføjes nye Access klasser. På samme måde er designet dependency inverted.

\begin{figure}[h]
	\centering
	\includegraphics[width=\linewidth]{figs/design/databaseERD_final_uml}
	\caption{Endeligt ER diagram, UML notation}
	\label{fig:databaseERD_final_uml}
\end{figure}

Connection serveren kan kalde sig ind i alt data-access lagets funktionalitet gennem ISmartpoolDB interfacet. I projektets dokumentation under \textit{Database og Data-access layer} beskrives det, hvorledes data-access laget bruges på \gls{windserver}.

\subsection{User story - Se sensor værdier}

I projektrapporten blev funktionalitet til forespørgsler og persistering af sensordata udviklet i 3. fase, hvor der allerede var en fuldt fungerende User- og Pool entitet og klasse. Se figur~\ref{fig:datasetentity}.

Som tidligere beskrevet er der lavet en DataAccess klasse til at lave data queries og persistering på databasen. Se klassediagrammet for DataAccess på figur~\ref{fig:dataAccess}.
	
\begin{figure}[h]
\centering
\includegraphics[width=0.7\linewidth]{figs/database/dataAccess}
\caption{Klassen DataAccess}
\label{fig:dataAccess}
\end{figure}

Igen blev \textit{Test Driven Development} brugt til dette, hvor koden blev skrevet til testene. Se dokumentation under afsnittet \textit{Test og Integration} for beskrivelse af tests.

Måden hvorpå der trækkes data ud af databasen er ved hjælp af LINQ queries. Brugen heraf er skrevet i detaljer i dokumentationen, under \textit{Database og Data-access layer}.

\begin{figure}[h]
	\centering
	\includegraphics[width=0.8\linewidth]{figs/implementering/datasetentity.png}
	\caption{Data Entity fra ER diagrammet i Visual Studio}
	\label{fig:datasetentity}
\end{figure}