\chapter{Metode}
I dette afsnit beskrives metoderne anvendt i forbindelse med udarbejdelsen af projektet.

\section{User Stories}
Til at specificere funktionelle krav for systemet bruges user stories \cite{margaretrouse2015}. User stories er en kort beskrivelse af en system-feature. En User Story er et brugsscenarie af systemet set fra slutbrugers perspektiv.

\section{FURPS}
FURPS er blevet brugt til analyse af systemets ikke-funktionelle krav. FURPS er et akronym, der beskriver forskellige kategorier af krav, i et system. Ved gennemgang af FURPS kategorierne, kunne relevante ikke-funktionelle krav bestemmes for vores system. FURPS er blevet brugt løbende igennem projektet, da projektarbejdet har været agilt.

\section{UML}
UML er brugt i vid udstrækning gennem alle projektets faser. UML danner grundstenen i projektets dokumentation, men har også hjulpet udviklingen undervejs. Visualiseringen har givet en bedre forståelse og gør det lettere at gennemskue nye og smartere designmuligheder. Blandt andet bruges lag-diagrammer til at beskrive systemarkitekturen, og sekvens- samt klassediagrammer til beskrivelse af softwaredesignet.



