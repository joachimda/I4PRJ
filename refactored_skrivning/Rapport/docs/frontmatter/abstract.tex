% Initial setup
\markboth{RESUMÉ/ABSTRACT}{RESUMÉ/ABSTRACT}
\abstractintoc
\abstractcol
\setlength{\abstitleskip}{-18pt}

%%%% RESUME BØR INDEHOLDE %%%%
% Projektets formål
% Projektets problemstillinger
% Valgte løsninger
% Anvendte metoder
% De væsentliseste resultater

% Dansk abstract
\begin{abstract}
%\lipsum[3-4] %% danske abstrakt skrives her
Dette 4. semester IKT projekt er lavet på ingeniørhøjskolen, Aarhus Universitet. Formålet med projektet er at kombinere faglige kompetencer fra semesterets kurser. Der er anvendt elementer fra Scrum, for at opnå erfaring med agil softwareudvikling. Arbejdet er foregået i sprints og med flad gruppestruktur

Der er udviklet et produkt, som kan monitorere vandforhold i en swimmingpool. Målingerne gemmes i en database. Dataoverførsel fra og til en database er sket gennem en server, som modtager forespørgsler fra en klient. Klienten er et GUI på brugerens PC eller smartphone og kan også tilgås via et website.

Gennem projektet er der skrevet automatiserede test, hvormed der løbende er ført kontrol med at koden virker efter hensigten. Git er brugt som versionsstyringsværktøj. 
\end{abstract}

% Engelsk abstract
\begin{abstracten}
%\lipsum[3-4] %% english abstract goes here
The following 4th semester ICT project is produced at Aarhus University, School of Engineering. The purpose has been to combine academic skills optained during the semester. Elements of Scrum has been applied to gain experience in agile software development. Work has been executed with a flat group hierarchy and completed in sprints.

A product which can monitor water conditions in a swimmingpool has been developed. Datatransfer to and from a database is executed by a server that responds to requests from a client. The client consists of a GUI on an users PC/smartphone and can also be accessed through a website.

During the project, automated tests have been written to perform ongoing control of developed code. Git has been used for version control. 
\end{abstracten}