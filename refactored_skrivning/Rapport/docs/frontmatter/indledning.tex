% !TEX root = ../../I4PRJ, Grp3 - Rapport.tex
\chapter{Indledning}

% Første afsnit af den egentlige rapport er indledningen, som skal give læseren den fornødne indføring i projektets emne, baggrund og formål. Undervurder derfor aldrig betydningen af dette afsnit.

% Indledningsafsnittet skal kunne besvare følgende spørgsmål: Hvad, Hvorfor og Hvordan. Hvad er emnet for rapporten, hvorfor man har valgt dette emne og hvordan man har tænkt sig at gennemføre opgaven.

% Indledningen kan også beskrive vigtige begreber, definitioner og anvendte forkortelser evt. angivet med en separat ordliste.

% Indledningen afsluttes med en læsevejledning, der giver en præsentation af rapportens opbygning.

Dette dokument er en projektrapport lavet af gruppe 3 i kurset I4PRJ4. Gruppen har haft Lars Mortensen som vejleder. 

Smartpool er et produkt, som kan hjælpe pool-ejere med at holde kontrol med poolen. Produktet kan overvåge vigtige pool-forhold, som sikrer de bedste og mest hygiejniske forhold. Det sikres ved hjælp af sensorer, som måler på forskellige forhold. Det er forhold såsom temperatur, luftfugtighed, pH og klorindhold. 

Systemet giver brugeren mulighed for, at få vist disse forhold via en applikation.

Projektet er valgt, da henvendelser til gruppen om et sådan produkt var givet på forhånd. 

Rapporten beskriver problemstillingen, som ønskes løst, hvilke dele som projektet løser og hvilke det ikke løser. I projektet specificeres kravene til systemet, hvordan systemet kan opbygges til at løse kravene, og hvordan kravene løses. 

Rapporten forklarer udviklingsmodellen for projektet og processen bag projektarbejdet. Rapporten samler op på projektets resultater og diskuterer løsninger på opståede problemstillinger og forslag til fremtidigt arbejde. 

% Please add the following required packages to your document preamble:
% \usepackage{multirow}
\begin{table}[]
\centering
\begin{tabular}{|l|l|l|}
\hline
Part                         & Layer            & Who               \\ \hline
\multirow{5}{*}{Application} & Windows GUI      & Alex \& Emil      \\ \cline{2-3} 
                             & iOS GUI          & Lasse             \\ \cline{2-3} 
                             & Web GUI          & Alex              \\ \cline{2-3} 
                             & Presenter        & Lasse             \\ \cline{2-3} 
                             & Model            & Lasse             \\ \hline
\multirow{4}{*}{Connection}  & .NET-client      & Joachim W.        \\ \cline{2-3} 
                             & iOS-client       & Lasse             \\ \cline{2-3} 
                             & Model            & Joachim W.        \\ \cline{2-3} 
                             & Server           & Joachim W.        \\ \hline
\multirow{2}{*}{Database}    & Data Acces Layer & Bjørn, Joachim A. \\ \cline{2-3} 
                             & Database         & Bjørn, Joachim    \\ \hline
\end{tabular}
\caption{Arbejdsfordelingstabel}
\label{table:arbejdsfordelingstabel}
\end{table}

Application: Arbejdet på Windows GUI påbegyndtes af både Alex og Emil. Det blev vurderet muligt at frigøre ressourcer til at implementere iOS og Web, fik Emil hovedansvaret for Windows GUI. Alex fik web og Lasse fik iOS. Selve applikationsdesignet er udarbejdet i fællesskab af alle tre.
