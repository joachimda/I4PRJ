{chapter}Opnåede erfaringer

%Beskrivelse af hvad man har lært ved at gennemføre projektet. Hvilke proble- mer man har haft undervejs og hvorledes de er blevet løst. I kan vælge at tilfø- je en individuel beskrivelse af egne opnået erfaringer og læring. Eventuelt en kort konklusion fra hver enkelt deltager om hvad hver deltager fik ud af pro- jektet (fx en kvart side per deltager).

%Når kurset er afsluttet, forventes den studerende at kunne:
%Udvælge og redegøre for en teknisk-faglig problemstilling som basis for projektet
% Anvende en iterativ udviklingsproces
% Dokumentere det udarbejdede produkt
%Anvende korrekt fagterminologi
%Udvikle applikationer med grafiske brugergrænseflader, databaser og netværkskommunikation
%Anvende teknikker, metoder og værktøjer til softwaretest
%Anvende objektorienteret analyse og design i systemudvikling
%Anvende projekt- og versionsstyringsværktøjer
%Kombinere viden fra flere af semestrets kurser og anvende denne i projektet.
%Diskutere og perspektivere proces-, design- og teknologivalg i projektet
%Udvælge, vurdere og anvende supplerende viden i projektarbejdet med angivelse af referencer.
%Præsentere projektets resultater ved et mundtligt forsvar

Gruppen har opnået erfaringer med den agile udviklingsproces scrum. Det iterative arbejde har medført højnet kvalitet på det opnåede resultat. Det skyldes bl.a. brugen af user stories, som egner sig bedre til agile processer og softwareudvikling generelt. Versionstyring af software har muliggjort parallelt arbejde i teamet og gjort test og integration til en iterativ proces. Parallelt arbejde har øget teamets effektivitet. \todo{Mere?}

Gruppen har erhvervet erfaringer med udvikling af GUI, databaser og netværkskommunikation. Der er udviklet GUI i både WPF og ASP.NET, som begge er dele af GUI kurset. Der er udviklet database og DAL med ADO.NET som undervises i DAB kurset. Til netværkkommunikation i systemet er udviklet en client-server arkitektur, som er en del af IKN kurset. \todo{problemer og løsninger}

Gruppen har opnået erfaringer med velovervejet strukturering af software arkitektur. Den velovervejede software arkitektur, hvor ansvar deles ud og afkobles har resulteret i et godt produkt. Det er eksempelvis lykkedes at implementerer en arkitektur, som supporterer multiplatform ved høj grad af afkobling. Det sikres ved brug af GUI arkitekturen MVP, som er pensum i SWD kurset. Implementeringen af web med client-server arkitekturen var kringlet og kunne muligvis forbedres med et Web API i stedet. Det er ikke undersøgt om resten af arkitekturen ligeledes ville forbedres ved samme ændring. Gruppen har haft fokus på design principper i alle designfaser. \todo{Har vi et eksempel.}

Gruppen har opnået erfaringer med softwaretest,  som dokumenterer sourcekoden. Test er skrevet i overenstemmelse med teorien far SWT kurset. Der er skrevet mange test, og der er sågar blevet testet op imod en database, som ikke er en del af SWT kurset. Gruppen har valgt ikke at unitteste GUI, da dette ej heller er en del af kurset. \todo{Eksempler problemer og løsninger}

Gruppen har fået erfaringer med kommunikation. Kommunikation og åbenhed har været i i fokus, da det i nogle tilfælde har introduceret små uoverensstemmelser. Uoverensstemmelserne er løst ved diskussion af problematikken. \todo{Eksempel der ikke udleverer.}

Gruppen har skrevet projektrapport og -dokumentation i Latex. Latex blev brugt i sammen med Git, som har muliggjort parallelt arbejde uden at skabe konflikter, når dokumenter gemmes. \todo{Hjælp}