\chapter{Fremtidigt arbejde}
Arbejde der skal laves i fremtiden, for at få videreudvikle Smartpool beskrives her.
\section{Udfærdigelse af projektet}
I dette afsnit beskrives hvad der skal gøres for, at produktet var klar til at blive installeret hos en kunde.

I applikationerne og på serveren skal der laves funktionalitet til, lave et nyt password, hvis brugeren trykker, at det gamle er glemt.
I applikationerne skal history view ændres, så brugeren har mulighed for at vælge hvor langt tilbage i tiden, han vil se data fra. På de andre tiers er det implementeret.

I Windows applikationen kunne alle kald til serveren laves asynkront, så UI ikke fryser i tilfælde af mangel på svar fra server eller database. Presenter laget burde også være klar til at modtage en fejlbesked i stedet for den forventede beskedtype.

De resterende user stories skal implementeres. F.eks. en brugers evne til at invitere en anden bruger til at blive "spectator" af hans pool, dvs. at se hans pool data uden evnen til at ændre det.

Websitets skal implementerer de to resterende view; 'EditPoolView' og 'HistoryView'. Controllers skal implementerer alle IView Interfaces, for derved at opnå den samme funktionaltet som de andre platforme.   

Server-applikationen i forbindelseslaget skal installeres og køre, på en reel server-computer med fast IP-adresse, således at brugere altid har mulighed for at tilgå den.

\todo{Fremtidigt arbejde. Alle skal lige læse den igennem og tænke over andre ting, der skal gøres.}

\section{Udviklingsperspektiver}
I projekt afgrænsningen er dataindsamling skåret fra dette projekt. For at den overordnede idé Smartpool er færdigt skal en Monitor unit og software dertil udvikles.

Det er meningen at systemet Smartpool skal færdiglaves og installeres i minimum en sommerhus pool.

Websitet har i ydermere mulighed for at fungere som kontaktside for produktet. Det er bl.a. salg af Smartpool-produkter samt download af software. 