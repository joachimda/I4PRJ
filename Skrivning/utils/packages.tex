% Encoding
\usepackage[utf8]{inputenc} % pdfLatex
\usepackage[T1]{fontenc}
\usepackage[danish]{babel}
\renewcommand{\danishhyphenmins}{22}
\usepackage[utf8]{inputenc} % æ ø å

% Date
%\usepackage[ddmmyyyy]{datetime}
%\renewcommand{\dateseparator}{.}

% Fonts
\usepackage{fourier}
\usepackage[scaled=0.8]{beramono}

% Math
\usepackage{amsmath,amssymb}
\usepackage{bm}
\usepackage{amsthm}
\usepackage{mathtools}

% Graphics
\usepackage[usenames,dvipsnames,table,xcdraw]{xcolor}
\usepackage{graphicx}
\usepackage{float}
%\usepackage[section]{placeins}
\usepackage{tikz}
\usepackage[pages=some]{background}
\usepackage{wrapfig}

% Listings & Tables
\usepackage{listings}
%\usepackage{pythontex}
\usepackage{enumitem}
\usepackage{tabu}
\usepackage{longtable}
\usepackage{multirow}
\usepackage{makecell}
\usepackage{tabularx}
\usepackage{booktabs}
\setlength{\tabcolsep}{10pt}
\renewcommand{\arraystretch}{1.2}

% References & Quotes
\usepackage[danish]{varioref}				% Muliggoer bl.a. krydshenvisninger med sidetal (\vref)
%\usepackage{nat}							% Udvidelse med naturvidenskabelige citationsmodeller
\usepackage[perpage]{footmisc}	% resets footnote counter on new page
\usepackage[danish=guillemets]{csquotes}
\usepackage[hidelinks]{hyperref}
\hypersetup{
    pdfstartview={FitH},
    pdftitle={Smart Pool 2.0},
    pdfsubject={Projektrapport},
    pdfauthor={I4PRJ4GRP3}
}
\usepackage[all]{hypcap}

% Etc
\usepackage{todonotes} % "to-do"-liste vha. \todo{opgave}
\usepackage[
	backend=biber,
	%backend=bibtex,
	style=ieee,
	natbib=true,
	backref=false,
	backrefstyle=all+,
	hyperref=true
]{biblatex}
\usepackage{pdflscape}
\usepackage{pdfpages}
\usepackage[nomain,toc,xindy,acronym,nonumberlist,noredefwarn]{glossaries}
\usepackage[xindy]{imakeidx}
\usepackage{float}
\makeindex

%=== Indstillinger til kodestykker ==========================%
\usepackage{listings} % kode
\lstset{
	language = C++, 
	backgroundcolor=\color{black!5},
	%basicstyle=\footnotesize,
	commentstyle=\color{mygreen},
	frame=single, 						% laver en ramme om koden
	keepspaces=true, 					% beholde indrykning
	keywordstyle=\color{blue},
	numbers=left, 
	numbersep=2pt, 						% afstand fra nummer til kode
	numberstyle=\color{mygray}, 		% linjenummer farve
	stringstyle=\color{green!70!black}, 		% string farve
	tabsize=5,
	captionpos=b, % sets the caption-position to bottom
	otherkeywords={null}, 	% til keywords 
	showstringspaces=false
}
\renewcommand\lstlistingname{Code listing}

\definecolor{bluekeywords}{rgb}{0,0,1}
\definecolor{greencomments}{rgb}{0,0.5,0}
\definecolor{redstrings}{rgb}{0.64,0.08,0.08}
\definecolor{xmlcomments}{rgb}{0.5,0.5,0.5}
\definecolor{types}{rgb}{0.17,0.57,0.68}

\usepackage{listings}
\lstset{language=[Sharp]C,
	captionpos=b,
	numbers=left, %Nummerierung
	numberstyle=\tiny, % kleine Zeilennummern
	frame=lines, % Oberhalb und unterhalb des Listings ist eine Linie
	showspaces=false,
	showtabs=false,
	breaklines=true,
	showstringspaces=false,
	breakatwhitespace=true,
	escapeinside={(*@}{@*)},
	commentstyle=\color{greencomments},
	morekeywords={partial, var, value, get, set},
	keywordstyle=\color{bluekeywords},
	stringstyle=\color{redstrings},
	%basicstyle=\ttfamily\small,
}

% Dummy
\usepackage{lipsum}