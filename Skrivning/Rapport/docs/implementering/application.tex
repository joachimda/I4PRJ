% !TEX root = ../../I4PRJ, Grp3 - Rapport.tex
\section{Applikationslaget}
Applikationslagets model- og presenterklasser er udviklet i C\#, i Microsoft Visual Studio. Klasserne er implementeret under følgende namespaces: Smartpool.Application.Model, Smartpool.Connection.Model og Smartpool.Application.Presentation. I dette afsnit beskrives de presenter-klasser, der blev introduceret i rapportens designafsnit. Herefter beskrives view-implementeringen i de platform-specifikke applikationer. For yderligere dokumentation af Smartpool.Application.Model og Smartpool.Application.Presentation henvises til dokumentationen.

\subsubsection{SignUpViewController.cs}
Implementeringen af SignUpViewController-klassen bestod hovedsageligt i at sammenbinde ISignUpViewController-interfacets metoder, med handlinger der skulle foretages i SignUpViewController-klassens ISignUpView. Listing~\ref{code:application_impl_signupviewc1} viser et eksempel på en sådan sammenbinding, hvor en tekstændring modtages fra view'et, gemmes i en af klassens properties, og view'et opdateres.

\begin{lstlisting}[caption={DidChangeNameText(...)},label={code:application_impl_signupviewc1}]
public void DidChangeNameText(string text)
{
	User.Name = text;
	UpdateSignUpButton();
}
\end{lstlisting}

Klassens SignUp metode er implementeret således, at den kontakter connection-serveren, med de oplysninger som presenteren har kendskab til, om den bruger der ønskes oprettet. Metoden kan ses i listing~\ref{code:application_impl_signupmethod}. Når et svar modtages fra serveren, tjekker presenteren hvorvidt anmodningen var succesfuld. Herefter kaldes en metode i view'et, alt efter om anmodningen gik igennem eller ej. Hvis anmodningen blev afvist kaldes DisplayAlert i view'et.

\begin{lstlisting}[caption={SignUp()},label={code:application_impl_signupmethod}]
public void SignUp()
{
	var signUpRequest = new AddUserRequestMsg(User.Name, User.Email, User.Passwords[0]);
	var response = _clientMessenger.SendMessage(signUpRequest);
	var generalResponse = (GeneralResponseMsg) response;

	if (generalResponse.RequestExecutedSuccesfully)
		_view.SignUpAccepted();
	else
		_view.DisplayAlert("Invalid Sign Up", response.MessageInfo);
}
\end{lstlisting}

\subsubsection{StatViewController.cs}
StatViewController-klassen håndterer indlæsning af pool-data til det tilknyttede view. I StatViewController-klassens ViewDidLoad, set i listing~\ref{code:application_impl_vdlstat}, bruges klassens PoolLoader til at indlæse pool-oplysninger i view'et.

\begin{lstlisting}[caption={ViewDidLoad() in StatViewController},label={code:application_impl_vdlstat}]
public void ViewDidLoad()
{
	// Load pools from server
	_loader.ReloadPools(_clientMessenger);

	// Load active pool info into text fields
	if (!_loader.PoolsAreAvailable())
		_view.DisplayAlert("No pools available", "You have not added any pools yet. Please go to 'Add Pool' to add a pool.");
	else
	{
		_view.SetAvailablePools(_session.Pools);
		_view.SetSelectedPoolIndex(_session.SelectedPoolIndex);
		LoadSensorData();
	}
}
\end{lstlisting}

ViewDidLoad-metoden benytter sig af en private hjælpemetode, LoadSensorData. Metoden, som ses i listing~\ref{code:application_impl_statlsd}, kalder view interface-metoden DisplaySensorData og bruger klassens PoolLoader, til at indlæse måleværdier i view'et.

\begin{lstlisting}[caption={LoadSensorData() in StatViewController},label={code:application_impl_statlsd}]
private void LoadSensorData()
{
	// Loads current sensor data into the view
	_view.DisplaySensorData(_loader.GetCurrentDataFromPool(_clientMessenger));
}
\end{lstlisting}

\subsection{Windows}
I Windows applikationen designes view-klasser, der implementerer view-interfacet defineret i præsentationslaget.
Designet af Windows GUI er lavet således, at codebehind filerne implementerer hver sit view-interfacet fra præsentationslaget. Codebehind agerer dermed som en bro, i mellem Smartpools præsentationslag, og WPF view-lag.

\subsection{iOS}
Implementeringen af iOS-applikationen består delvist i den visuelle implementering af brugergrænsefladen, og den kode-mæssige implementering af de bagomliggende view-klasser. Som vist i designafsnittet, implementeres view-klasserne som controller-broer, der implementerer view-interfaces angivet i præsentationslaget. Implementeringen af view-klasserne foretages i udviklingsværktøjet Xamarin, i sproget C\#. Den visuelle implementering foretages med storyboards i Interface Builder, som er en del af Xcode-udviklingsværktøjet.

\subsubsection{StatViewBridge.cs}
StatViewBridge implementerer IStatView-interfacet. StatViewBridge nedarver, i modsætning til de statiske views, fra UIViewController sub-klassen UITableViewController. Denne UIKit controller-klasse kan bruges til at vise lister der ændre sig i runtime, som er nyttigt for StatViewBridge-klassen, da den skal kunne opdateres med måledata. I listing~\ref{code:ios_impl_statdisplay} ses implementeringen af interface-metoden DisplaySensorData. I denne metode gemmes den modtagne sensor data, så den kan bruges som data source til UITableViewController-listen.

\begin{lstlisting}[caption={DisplaySensorData(...)},label={code:ios_impl_statdisplay}]
public void DisplaySensorData(List<Tuple<SensorTypes, double>> sensorData)
{
	_sensorData = sensorData;
	_sensorData.Sort ();
	TableView.ReloadData ();
}
\end{lstlisting}

Klassen overrider en række af UITableViewController-klassens metoder, der indlæser data i view'et. De overrides der bestemmer indholdet af listens elementer, fremgår af listing~\ref{code:ios_impl_stattableor}. I GetCell-metoden opsættes en StatViewCell (StatViewCell.xib), til at vise den data der er modtaget fra presenter-klassen.

\begin{lstlisting}[caption={Overrides af UITableViewController-metoder i StatViewBridge},label={code:ios_impl_stattableor}]
public override nint RowsInSection (UITableView tableView, nint section)
{
	return _sensorData.Count;
}

public override UITableViewCell GetCell (UITableView tableView, Foundation.NSIndexPath indexPath)
{	
	var cell = tableView.DequeueReusableCell (_reuseIdentifier) as StatViewCell;
	var type = string.Format ($"{_sensorData [indexPath.Row].Item1}");
	ell.DataLabel.Text = string.Format ($"{_sensorData [indexPath.Row].Item2}") + GuiCharacter.SignForType(_sensorData [indexPath.Row].Item1);
	cell.NameLabel.Text = type;
	cell.BorderImage.Image = UIImage.FromFile (type.ToLower () + ".png");
	return cell;
}
\end{lstlisting}