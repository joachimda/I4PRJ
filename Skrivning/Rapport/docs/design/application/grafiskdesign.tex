% !TEX root = ../../../I4PRJ, Grp3 - Rapport.tex
\chapter{Grafisk Design}
I dette afsnit beskrives det grafiske design af brugergrænsefladen. Det grafiske design skal afspejle de respektive views funktionalitet. Først udarbejdes et konceptuelt design. GUI views designes ved at analyserer og diskuterer user stories for systemet. Resultatet af analysen er et klassediagram for view'et, som ses i de tidligere design afsnit. Et håndtegnet design med tilhørende kan så laves ud fra klassediagrammet og dets attributter.
Analyseresultatet dannede grundstenene for GUI folkenes arbejde. Designet blev dermed ensartet for de forskellige platforme. Det fælles udarbejdede design har mindsket en muligt senere kommende bureaukratisk proces, da alle udviklere ved at målet er nået, når GUI svarer til design. 

\section{Konceptuel Design}
Først udarbejdes et konceptuelt design. Som eksempel er user story omhandlende login behandlet og kan ses på figuren nedenfor:

\begin{figure}
	\centering
	\includegraphics[width=0.5\linewidth]{figs/design/concuptuel_design_loginview}
	\caption{Konceptuelt design af LoginView}
	\label{fig:conceptualdesignview}
\end{figure}

De konceptuelle design udvikles til et grafisk design.

\begin{figure}
	\centering
	\includegraphics[width=0.5\linewidth]{figs/design/DesktopHDLogin}
	\caption{Grafisk design af LoginView}
	\label{fig:graphicaldesign}
\end{figure}
Udviklere af views til de forskellige applikationer stræber efter at opnå et visuelt design der ligner dette mest muligt.