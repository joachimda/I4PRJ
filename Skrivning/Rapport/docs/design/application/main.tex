\section{Applikationslaget}
Applikationslaget er det lag af software i det samlede distribuerede system, som installeres eller afvikles hos brugeren. Applikationslaget er designet til at være platformuafhængigt, jf. systemets arkitektur. Derved er lagets model- og præsentationslag designet til at være portabelt, hvorimod view-laget er platformspecifikt. 

Som følge af multi-lag arkitekturen kunne applikationslaget samt de platform-specifikke view-implementeringer
designes, uafhængigt af hinanden, så længe interfaces blev specificeret undervejs i processen. 

Presenter interfacet specificerer hvilke metoder og hvilken funktionalitet et view skal have.
Her ses klassediagrammet for ISignUpView.

\begin{figure}
\centering
\includegraphics[width=0.35\linewidth]{figs/design/application_isignupview}
\caption{ISignUpView}
\label{fig:application_isignupview}
\end{figure}
View'et i hver applikation implementerer interfacet og opretter en controller, hvis interface ses nedenfor.
Interfacet indeholder de funktioner user stories'ne leder op til.
\begin{figure}
\centering
\includegraphics[width=0.7\linewidth]{figs/design/application_signupviewcontrollerandinterface}
\caption{ISignUpViewController og SignUpViewController}
\label{fig:application_isignupviewcontroller}
\end{figure}
Figur~\ref{fig:application_isignupviewcontroller} viser også SignUpViewController's konkrete klasse, der indeholder flere metoder og attributter, som måden MVP er lavet i applikationslaget viste sig at kræve.
Det er attributter som \_clientMessenger, som er klienten, der kommunikerer med serveren. 
\_user af typen UserValidater er en klasse i præsentationslaget der er platformsuafhængig til at validere om bruger informationen er gyldig. 

Controlleren kender ISignUpView, så den kalder funktionerne i view'et og på den måde er logik og view adskilt.

Designet af funktionaliteten bag user story'en, der viser nyeste målinger har ført til følgende konkrete controller klasse.

\begin{figure}
\centering
\includegraphics[width=0.35\linewidth]{figs/design/application_statviewcontroller}
\caption{StatViewController konkret klasse}
\label{fig:application_statviewcontroller}
\end{figure}

For yderligere forklaring se dokumentation afsnit Applikationslaget under Design.

\subsection{Windows GUI}
I Windows applikationen designes view-klasser, der implementerer view-interfacet defineret i præsentationslaget.
Designet af Windows GUI er lavet således, at codebehind filerne implementerer hver sit view-interfacet fra præsentationslaget. Codebehind agerer dermed som en bro, i mellem Smartpools præsentationslag, og WPF view-lag.

I klasse diagrammet nedenfor, ses Windows designet, af WinCreateUserView der implementerer ISignUpView fra applikationslaget og har en SignUpViewController.
\begin{figure}
\centering
\includegraphics[width=0.7\linewidth]{figs/design/wincreateuserandwinstatviewview}
\caption{WinCreateUserView og WinStatView}
\label{fig:wincreateuserandwinstatviewview}
\end{figure}

Ligeledes er view klassen for WinStatView designet.
Klassen ses på figur~\ref{wincreateuserandwinstatviewview}.

For yderligere forklaring se dokumentation afsnit Applikationslaget under Design.

% !TEX root = ../../../I4PRJ, Grp3 - Rapport.tex
\chapter{Grafisk Design}
I dette afsnit beskrives det grafiske design af brugergrænsefladen. Det grafiske design skal afspejle de respektive views funktionalitet. Først udarbejdes et konceptuelt design. GUI views designes ved at analyserer og diskuterer user stories for systemet. Resultatet af analysen er et håndtegnet design med tilhørende noter fra diskussionerne. Analyseresultatet dannede grundstenene for GUI folkenes arbejde. Designet blev dermed ensartet for de forskellige platforme. Det fælles udarbejdede design har mindsket en muligt senere kommende bureaukratisk proces, da alle udviklere ved at målet er nået, når GUI svarer til design. 

\section{Konceptuel Design}
Først udarbejdes et konceptuelt design. Det konceptuelle design udarbejdes ved analyse og diskussion af user stories, som eksempel er user story omhandlende login behandlet og kan ses på figuren nedenfor:

\begin{figure}
	\centering
	\includegraphics[width=\linewidth]{figs/design/concuptuel_design_loginview}
	\caption{Domænemodel for systemet}
	\label{fig:domainmodel}
\end{figure}

De konceptuelle design udvikles

De forskellige platformes GUI minder i så høj grad om hinanden, at hvert view og hvilke user stories de implementerer kun vil blive beskrevet for en platform. 