% Initial setup
\markboth{RESUMÉ/ABSTRACT}{RESUMÉ/ABSTRACT}
\abstractintoc
\abstractcol
\setlength{\abstitleskip}{-18pt}

%%%% RESUME BØR INDEHOLDE %%%%
% Projektets formål
% Projektets problemstillinger
% Valgte løsninger
% Anvendte metoder
% De væsentliseste resultater

% Dansk abstract
\begin{abstract}
%\lipsum[3-4] %% danske abstrakt skrives her
Dette 4. semester IKT projekt er lavet på ingeniørhøjskolen, Aarhus Universitet. Formålet med projektet er at kombinere faglige kompetencer fra semesterets kurser. Der er anvendt elementer fra Scrum, for at opnå erfaring med agil softwareudvikling i en iterativ udviklingsproces. Arbejdet er foregået i sprints og med flad gruppestruktur

Der udviklet et produkt, som kan monitorere vandforhold i en swimmingpool. Målingerne gemmes så i en database. Dataoverførsel fra og til databasen er sket gennem en server, som modtager forespørgsler fra en klient. Klienten er en GUI brugerens PC eller smartphone.

Gennem projektet er der skrevet automatiserede test, hvormed der løbende er ført kontrol med at koden virker efter hensigt. Git er brugt som versionsstyringsværktøj. 
\end{abstract}

% Engelsk abstract
\begin{abstracten}
\lipsum[3-4] %% english abstract goes here
\end{abstracten}