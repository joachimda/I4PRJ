\chapter{Udviklingsværktøjer}

I gennem dette semesterprojekt er der blevet brugt en lang række udviklingsværktøjer. Mange af værktøjerne er både blevet brugt i designfaserne og senere til implementering af systemet. Brugen af de forskellige værktøjer har været baseret på, hvilket emne gruppen har arbejdet med.

\section*{DDS-lite}
DDS-lite er brugt i kurset I4DAB\footnote{Databasefag.}, har været brugt til at designe de første udkast til en database. DDS-lite giver mulighed for at autogenerere SQL scripts som kan køres på databasen, hvorved tabeller med mere oprettes.

\section*{Microsoft Visual Studio 2015}
Er anvendt til udvikling af source code, database og enkelte diagrammer
% nuget pakker
% andre versioner?

\section*{TeXstudio}
Er anvendt til skrivning af rapport og dokumentation. Dette blev valgt fremfor lignende programmer så som Word, da TeXstudio giver en mere ensartet struktur, da designvalg, så som teksttype, størrelse osv, foretages i en separat fil
% miktek compiler

\section*{Git}
Er anvendt til versionsstyring af source code, samt rapport og dokumentation.
% github
% git bash
% Atlassian SourceTree

\section*{Microsoft Visio 2013}
Microsoft Visio 2013 er et af de værktøjer, som er brugt tidligt i projektet. Visio er blevet anvendt til at designe SysML-diagrammer, så som blokdiagrammer og applikationsmodeller, samt at oprette UML klassediagrammer. Generelt har dette værktøj været det mest udbredte, da alle dele af semesterprojektet skulle designes og beskrives.

\section*{Enterprise Architect}
Enterprise Architect er et program, som bruges til at lave diagrammer, det er især brugt til at lave klasserdiagrammer med mere, for den skrevne kode. En særlig feature som er brugt meget, er Enterprise Architect's mulighed for automatisk at generere diagrammer udfra rå source kode.

\section*{Dropbox}
Er anvendt til deling af dokumenter i rapporten

\section*{ReSharper} % skal den med?
Er anvendt som en del af Visual Studio. ReSharper medvirker til at give et forbedret udviklingsmiljø, og en mere ensartet kode 

\section*{NUnit}
Er et test framework som er anvendt til at skrive automatiske tests

\section*{NSubstitute}
Er et mocking bibliotek der anvendes til at generere fakes til tests i NUnit

\section*{Entity Framework}
Er et framework der giver mulighed for at arbejde med objekt orienteret programmering, indenfor udvikling af relationelle databaser. 

\section*{Pivotal Tracker}
Er anvendt som en form for scrum board, hvor user stories er oprettet og har haft en status som f.eks. ongoing. 
