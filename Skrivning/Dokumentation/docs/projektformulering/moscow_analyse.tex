\section{MoSCoW analyse}
\gls{moscow}\footnote{\url{https://en.wikipedia.org/wiki/MoSCoW_method}} analyse er udarbejdet efter principperne bag \gls{moscow} er en prioriteringsteknik brugt i projektstyring. Systemet har følgende prioriteringer:

\subsection*{Must}
\begin{itemize}
	\item Der skal kunne oprettes en bruger til systemet.
	\item Systemets data skal gemmes i en database.  
	\item Brugeren skal kunne tilføje minimum én pool til sin konto. 
	\item Brugeren skal kunne interagere med systemet via en \gls{gui}. 
	\item Brugeren skal kunne se sine nuværende data.
\end{itemize}

\subsection*{Should}
\begin{itemize}
	\item Der bør kunne tilføjes flere pools til en bruger. 
	\item Systemet bør kunne vise grafer over brugerens data.
	\item Brugeren bør kunne se gamle data.
	\item Brugeren bør kunne indstille ønsket område for værdier.
	\item Systemet bør have fastlagte værdier for tilladte niveauer af ph og klor, så brugeren ikke kan indstille ønsket måleværdi til en ikke lovlig værdi.\todo{Indsæt reference til dette reglementt.}
	\item Brugeren bør kunne indtaste data om sin pool.
	\item Systemet bør kunne udføre beregninger på tilføjelse af diverse kemikalier.\todo{Skal omformuleres.}
\end{itemize}

\subsection*{Could}
\begin{itemize}
	\item Kan indeholde en app til iOS.
	\item Kan indeholde en \gls{webapp}.
	\item Kan notificerer brugeren.
	\item Brugeren kan slå notifikationer fra.
	\item Kan interagere med kundes pool system.
\end{itemize}

\subsection*{Wont}
\begin{itemize}
	\item Systemet kommer ikke til at få reelle data. Inputs vil være virtuelle.
\end{itemize}
