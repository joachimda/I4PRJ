\chapter{Software Test og Integration}

% indledning til kapitel
Som en del af gruppens måde at bruge scrum på er der løbende integreret. Omkring hver anden uge har der været en målsætning om at have en fase af systemet færdig. 

System er udviklet i tre faser: 

\begin{enumerate}
	\item \textbf{Brugerfunktionalitet}\\
	En Bruger skal kunne oprettes, fjernes og findes. Herunder ligger også funktionalitet til validering af email og password. 
	\item \textbf{Poolfunktionalitet}\\
	Ligeledes skal en pool kunne sættes ind i systemet. En Bruger skulle i denne faser kunne være tilknyttet flere pools.
	\item \textbf{Datafunktionalitet}\\
	Sensormålinger skal kunne opbevares og vises. 
\end{enumerate}

Efter hver af disse faser hvor henholdsvis databasen, \gls{windserver} og GUI applikationerne.

% værktøjer til test... Nunit, Nsub, github, travis?

% Siden første design spike er hver eneste testede ændring blevet først ind i kodebasen som alle har brugt... på denne måde er systemet gennem hele udviklingsforløbet holdt opdateret og klar til release...

% find et dependency diagram og sæt ind her, og henvis til i dette afsnit

% generelt er systemet integrationstestet 'bottom-up' idet at det var nemmest at sikre at databasen virkede pga mange nemme enhedstest, dernæst kunne connection serveren kobles på, hvorefter applicationen kunne sættes sammen med serveren.

% hvordan integrationen er designet for hhv. db, connection og gui... afhængigheder, problemer... hvordan det er gået med at teste det løbende

%\section{Database og Data Access Layer}

Der er med udgangspunkt i designovervejelserne i afsnit~\ref{sec:designdatabase} implementeret et fungerende data-access layer med tilhørende database.

Databasen er implementeret med en Model First tilgang \cite{microsoftdatadevelopercenter2016}. Det vil sige at der opsættes en model (ER diagram) for databasen i Visual Studio, hvorefter der genereres et SQL script der kan køres mod den specifikke database. Scriptet køres mod en tom database, hvor de opstillede entities genereres som tabeller.

Data entiteten som ses på figur~\ref{fig:databaseERD_final_uml} da bruges som en klasse, se klassedigram på figur~\ref{fig:efGeneratedData}. De forskellige datatyper, pH, chlorine, temperature og humidity figurerer som lister (ICollections) i Data klassen. Disse lister er i koden angivet som virtual. Dette er for at gøre lazy loading muligt.

\begin{figure}
\centering
\includegraphics[width=0.5\linewidth]{figs/implementering/efGeneratedData.PNG}
\caption{Data klassen - Genereret fra Entity Model}
\label{fig:efGeneratedData}
\end{figure}

\subsection{Implementering af data-access layer}

\subsubsection{Træk pooldata ud}

\begin{lstlisting}[caption= GetChlorineData method, label=code:getChlorineData]

public List<Tuple<SensorTypes, double>> GetChlorineValues(string poolOwnerEmail, string poolName, int daysToGoBack)
{
double days = System.Convert.ToDouble(daysToGoBack);
string now = DateTime.UtcNow.ToString("G");
string start = DateTime.Parse(now).AddDays(-days).ToString("G");

using (var db = new DatabaseContext())
{   
DateTime startTime = DateTime.ParseExact(start, "dd/MM/yyyy HH:mm:ss", System.Globalization.CultureInfo.InvariantCulture);
DateTime endTime = DateTime.ParseExact(now, "dd/MM/yyyy HH:mm:ss", System.Globalization.CultureInfo.InvariantCulture);

var chlorineDataQuery = from chlorine in db.ChlorineSet
where chlorine.Data.Pool.Name == poolName && chlorine.Data.Pool.User.Email == poolOwnerEmail
select chlorine;

List<Tuple<SensorTypes, double>> chlorineTuples = new List<Tuple<SensorTypes, double>>();

foreach (var chlorine in chlorineDataQuery)
{
if(DateTime.Parse(chlorine.Data.Timestamp).CompareTo(endTime) < 0 ||
DateTime.Parse(chlorine.Data.Timestamp).CompareTo(startTime) > 0)
{
chlorineTuples.Add(new Tuple<SensorTypes, double>(SensorTypes.Chlorine, chlorine.Value));
}
}

return chlorineTuples;
}
}
\end{lstlisting}

\subsubsection{Tilføj user}

\begin{lstlisting}[]
public bool AddUser(string fullname, string email, string password)
{

	if (IsEmailInUse(email)) return false;

	User user;

	if (!ValidateName(fullname)) return false;

	string[] names = fullname.Split(' ');

	if (names.Length <= 2)
	{
		user = new User() { Firstname = names[0], Lastname = names[1], Email = email, Password = password };
	}
	else
	{
		user = new User() { Firstname = names[0], Middelname = names[1], Lastname = names[2], Email = email, Password = password };
	}

	using (var db = new DatabaseContext())
	{
		db.UserSet.Add(user);
		db.SaveChanges();
	}

	return true;
}
\end{lstlisting}

%\subsection{Implementering}
I følgende afsnit vil den overordnede implementering af de enkelte dele blive beskrevet.
\subsubsection{Overordnet connection struktur}
\textit{Klient} delen består af et model projekt som er generelt for alle platforme, samt et klient projekt som er specifikt for hver platform. Dette er gjort for at gøre så meget som muligt anvendeligt på alle platforme. I model projektet er desuden defineret en række besked objekter, som anvendes ved kommunikation mellem klient og server.

\textit{Server} delen består af en række systemer som tilsammen udgør en samlet server udviklet til at køre på en windows pc. Al kommunikation til databasen foregår fra server delen.

\subsubsection{Klient}
Klienten modtager besked objekter fra applikations laget, og omdanner disse til en streng vha. Json serializering. Strengen bliver sendt til server delen gennem en socket klient der passer til den pågældende platform.
Klienten modtager derefter et svar fra serveren, i form af et serializeret besked objekt, som bliver deserializeret til basis besked klassen. Denne indeholder en besked type, og klienten kan derefter deserializere den modtagne streng til det korrekte besked objekt. Derefter bliver objektet sendt videre, tilbage til applikations laget. 

\subsubsection{Server}
Vedligeholder følgende funktioner i systemet
\begin{itemize}
	\item Modtage, behandle og svare på requests fra klient delen
	\item Varetage user sessions
	\item Kommunikere med databasen via metoder i database delen
	\item Simulere pool data
\end{itemize}

\subsubsection{Modtage, behandle og svare på requests fra klient delen}
Serveren modtager via en Asyncronous Socket Client (ASC) en streng fra klient delen. Denne bliver, som i klienten, lavet til et basis besked objekt. Denne bliver derefter behandlet vha. en switchcase, hvor der reageres på hvilken message type der er blevet sendt. Dette foregår i en ResponseManager klasse, og denne vil, i nogle situationer, være i stand til selv at udføre den kaldte request. Det gør den ved f.eks. at lave et kald til databasen og returnere svaret derfra. 
I andre situationer, bliver kaldet sendt videre til en sub handler, som f.eks. TokenMsgResponse, der tager sig af alle requests, som kræver at brugeren er logget ind. Dette er lavet således at ResponseManager checker, via databasen, om brugeren er logget ind, og hvis dette er tilfældet, bliver beskeden sendt videre til TokenMsgResponse. TokenMsgResponse behøver dermed ikke selv at kontrollere om brugeren er logget ind.
Hvis beskedtypen ikke genkendes, sendes svar tilbage til klienten om dette.

\subsubsection{User sessions}
For at holde styr på hvilke brugere som er logget ind, er der udviklet et Token system. Dette system giver en øget sikkerhed, ved at brugeren kun sender password en enkelt gang, og det behøver derfor heller ikke at blive gemt i klienten. Desuden bliver der færre kald til databasen, da efterfølgende requests ikke behøver at kontrollere brugerens password via databasen.
Token systemet virker ved at en bruger, ved login, får tilknyttet en streng af tilfældige karakterer. Brugernavnet bliver, sammen med strengen af karakterer og et timestamp, gemt i en klasse der hedder Token. Alle disse tokens bliver så vedligeholdt i en TokenKeeper. Når en bruger efterfølgende laver en request til serveren, sender klienten både username og token strengen med i sin request. Serveren kontrollerer derefter om dette stemmer overens med de data som ligger i TokenKeeperen, samt om det gemte timestamp er ældre end systemets valgte sessions tid.
En bruger kan godt være logget ind på flere enheder på samme tid. I det tilfælde vil hver enhed få tildelt en token streng, og dermed har de ikke indflydelse på hinandens session.

\subsubsection{Kommunikere med databasen via metoder i database delen}
Serveren kommunikerer via tilgængelige metoder i systemets dataaccess layer ??

\subsubsection{Simulerering af pool data}
Da Smartpool systemet ikke anvender reele data, er der udviklet en FakePool klasse til at simulere dette. Denne opretter en af hver type sensor. Hver sensor bliver initieret med en værdi, der ligger indenfor et realistisk område, for den pågældende sensor type. 
Denne værdi opdateres med et angivet internal, hvorefter værdierne gemmes i databasen. Værdi opdateringen foregår med små tilfældigt genererede ændringer. Disse er yderligere begrænset af en minimum og maximum grænse, specificeret i en SensorValueAuthenticator klasse.  

%% !TEX root = ../../I4PRJ, Grp3 - Rapport.tex
\subsubsection{Implementering}
Applikationslagets model- og presenterklasser er udviklet i C\#, i Microsoft Visual Studio. Klasserne er implementeret under følgende namespaces: Smartpool.Application.Model, Smartpool.Connection.Model og Smartpool.Application.Presentation. I dette afsnit beskrives de presenter-klasser, der blev introduceret i rapportens designafsnit. Herefter beskrives view-implementeringen i de platform-specifikke applikationer. For yderligere dokumentation af Smartpool.Application.Model og Smartpool.Application.Presentation henvises til dokumentationen.

\paragraph{SignUpViewController.cs}
Implementeringen af SignUpViewController-klassen bestod hovedsageligt i, at sammenbinde ISignUpViewController-interfacets metoder, med handlinger der skulle foretages i SignUpViewController-klassens ISignUpView. En af disse handlinger er oprettelsen af en bruger. Metoden SignUp, der kaldes når presenter-klassen modtager et ButtonPressed kald fra view'et, kan ses i listing~\ref{code:application_impl_signupmethod}. I metoden oprettes en besked, der sendes til serveren gennem den presenterens IClientMessenger. Når et svar modtages fra serveren, tjekker presenteren hvorvidt anmodningen var succesfuld. Herefter kaldes en metode i view'et, alt efter om anmodningen gik igennem eller ej. Hvis anmodningen blev afvist kaldes DisplayAlert i view'et.

\begin{lstlisting}[caption={SignUp()},label={code:application_impl_signupmethod}]
public void SignUp()
{
	var signUpRequest = new AddUserRequestMsg(User.Name, User.Email, User.Passwords[0]);
	var response = _clientMessenger.SendMessage(signUpRequest);
	var generalResponse = (GeneralResponseMsg) response;

	if (generalResponse.RequestExecutedSuccesfully)
		_view.SignUpAccepted();
	else
		_view.DisplayAlert("Invalid Sign Up", response.MessageInfo);
}
\end{lstlisting}

\paragraph{StatViewController.cs}
StatViewController-klassen håndterer indlæsning af pool-data til det tilknyttede view. I StatViewController-klassens ViewDidLoad, set i listing~\ref{code:application_impl_vdlstat}, bruges klassens PoolLoader til at indlæse pool-oplysninger i view'et.

\begin{lstlisting}[caption={ViewDidLoad() in StatViewController},label={code:application_impl_vdlstat}]
public void ViewDidLoad()
{
	// Load pools from server
	_loader.ReloadPools(_clientMessenger);

	// Load active pool info into text fields
	if (!_loader.PoolsAreAvailable())
		_view.DisplayAlert("No pools available", "You have not added any pools yet. Please go to 'Add Pool' to add a pool.");
	else
	{
		_view.SetAvailablePools(_session.Pools);
		_view.SetSelectedPoolIndex(_session.SelectedPoolIndex);
		LoadSensorData();
	}
}
\end{lstlisting}

ViewDidLoad-metoden benytter sig af en private hjælpemetode, LoadSensorData. Metoden, som ses i listing~\ref{code:application_impl_statlsd}, kalder view interface-metoden DisplaySensorData og bruger klassens PoolLoader, til at indlæse måleværdier i view'et.

\begin{lstlisting}[caption={LoadSensorData() in StatViewController},label={code:application_impl_statlsd}]
private void LoadSensorData()
{
	// Loads current sensor data into the view
	_view.DisplaySensorData(_loader.GetCurrentDataFromPool(_clientMessenger));
}
\end{lstlisting}


